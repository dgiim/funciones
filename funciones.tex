\documentclass[a4paper, 11pt]{amsart}
\usepackage{amssymb}
\usepackage{amsthm}
\usepackage{enumerate}
\usepackage[hidelinks]{hyperref}

%%% Castellano
\usepackage[spanish,es-noquoting,es-lcroman]{babel} 
\selectlanguage{spanish}
\usepackage[utf8]{inputenc}
%% Multicolumns
\usepackage{multicol}
%% Empty set
\let\emptyset\varnothing
\listfiles
%%% Twopart definition
\newcommand{\twopartdef}[4]
{
	\left\{
		\begin{array}{ll}
			#1 & \mbox{si } #2 \\
			#3 & \mbox{si } #4
		\end{array}
	\right.
}


%% Theorem environments
\newtheorem{theorem}{Teorema}[section]
\newtheorem{lemma}[theorem]{Lema}

\theoremstyle{definition}
\newtheorem{definition}[theorem]{Definición}
\newtheorem{example}[theorem]{Ejemplo}
\newtheorem{exca}[theorem]{Ejercicio}

\theoremstyle{remark}
\newtheorem{remark}[theorem]{Remark}

\numberwithin{equation}{section}

\everymath{\displaystyle} 





\begin{document}

\title{Funciones con nombre propio}

%    Remove any unused author tags.
%    author one information
\author{Ignacio Cordón}
\address{}
\curraddr{}
\email{}
\thanks{}

% References
%\subjclass[2000]{Primary }
%    For articles to be published after 1 January 2010, you may use
%    the following version:
%\subjclass[2010]{Primary }

\keywords{}

\date{}

\dedicatory{}


\begin{abstract}
  Definición y propiedades fundamentales de algunas funciones elementales.
\end{abstract}

\maketitle

  \section{Función gamma}
  
  \begin{definition}
  Para $0 < x < \infty$, se define la funcion gamma como: 
    \begin{equation}
      \Gamma(x) = \int_0^{\infty}{t^{x-1}e^{-t}dt}
      \label{gamma}
    \end{equation}
  \end{definition}
  
  Es sabido que esta integral converge para cualquier valor de $x\in]0,+\infty[$, y que por tanto la definición de 
  $\Gamma$ es correcta.\\
  
  \begin{theorem}
    Se verifica:
      \begin{enumerate}[(i)]
	\item $\Gamma(x+1) = \Gamma(x)$
	\item $\Gamma(n+1) = n!$ para cualquier $n\in \mathbb{N}$
	\item $\log \circ \Gamma$ es convexa en $]0,+\infty[$
      \end{enumerate}
  \end{theorem}
  
   \begin{proof}

     \begin{enumerate}[(a)]
      \item Integrando por partes tenemos:
	\begin{eqnarray*}
	\Gamma(x+1) &=& \int_0^{\infty}{t^{x}e^{-t}dt} = \bigg[-e^{-t}t^x\bigg]_0^{\infty} +
	  \int_0^{\infty}{xt^xe^{-t}dt} = \\
	  & = & x\int_0^{\infty}{t^{x-1}e^{-t}dt} = x\Gamma(x)
	\end{eqnarray*}
      \item Inducción sobre $n$, usando (a) y $\Gamma(1) = \int_0^{\infty}{e^{-t}dt} = 
	\bigg[-e^{-t}\bigg]_0^{\infty} = 1 $
      \item Dados $p,q \in \mathbb{N} : \frac{1}{p} + \frac{1}{q} = 1$\\
      
	La \textit{desigualdad de Young} para $a,b \ge 0$ nos dice: 
	  $$ab \le \frac{a^p}{p} + \frac{b^q}{q}$$\\
	La \textit{desigualdad de Holder} nos dice: 
	  $$\bigg|\int_a^b {fg dt}\bigg| \le 
	  \bigg(\int_a^b {|f|^p dt}\bigg)^{1/p} \bigg(\int_a^b {|g|^p dt}\bigg)^{1/q}$$
	  
	Así:
	
	\begin{eqnarray*}
	 \Gamma\bigg(\frac{x}{p} +\frac{y}{q}\bigg) & = & 
	    \int_0^{\infty}{t^{\big(\frac{x}{p}+\frac{y}{q}
	    -\frac{1}{p}-\frac{1}{q}\big)}e^{-t}dt} 
	    =\int_0^{\infty}{t^{\big(\frac{1}{p}(x-1)\big)} t^{\frac{1}{q}(y-1)\big)}e^{-t}dt} =\\
	    \\
	 & = &_{(*)} \le \Gamma(x)^{1/p}\Gamma(x)^{1/q}
	\end{eqnarray*}\\
	
	Donde en $(*)$ se ha usado la desigualdad de Holder anteriormente descrita.

     \end{enumerate}
      
    \end{proof}
  
  De hecho estas tres propiedades caracterizan a la función $\Gamma$, como demostraron Bohr y
  Mollerup en el teorema al que dan nombre:
      
  \begin{theorem}{Caracterización de Bohr-Mollerup}\\
  
  Sea $f\ge 0$ sobre $]0,+\infty[$ verificando:
  
  \begin{enumerate}[(a)]
   \item $f(x+1) = xf(x)$
   \item $f(1) = 1$
   \item $\log \circ f es convexa$
  \end{enumerate}
  
   Entonces $f(x) = \Gamma(x)$
  \label{caracterizacion}
  \end{theorem}

  \begin{proof}
   Veremos que $f(x)$ queda unívocamente determinada, ya que $\Gamma$ verifica las hipótesis del teorema.
   Por a), basta hacerlo para $x\in]0,1[$. Llamamos $\varphi = log \circ f$:
   
   \begin{equation}
    \varphi (x+1) = \varphi(x) + log(x), \forall x\in]0,+\infty[
    \label{phi}
   \end{equation}
   
   $\varphi(0) = 1$ y $\varphi$ es convexa. Suponemos en lo que sigue $ x\in]0,+\infty[$ y $n\in \mathbb{N}$.
   Claramente $\varphi(n+1) = log(n!)$. Además, por \ref{phi}:
   
   \begin{eqnarray*}
    \varphi(n+1)-\varphi(n) = log(n)\\
    \varphi(n+2)-\varphi(n+1) = log(n+1)
   \end{eqnarray*}
   
   y por convexidad:
   \begin{eqnarray*}
    \varphi (n+1+x) &=& \varphi(x(n+2) +(1-x)(n+1)) \le x\varphi(n+2) + (1-x)\varphi(n+1) \Leftrightarrow\\
    &\Leftrightarrow& \varphi (n+1+x) - \varphi(n+1) \le x\big(\varphi(n+2)-\varphi(n+1)\big)\\
    \\\
    \varphi(n+1) &=& \varphi(x(n+x)+(1-x)(n+x+1)) \le x\varphi(n+x) + (1-x) \varphi(n+x+1) \Leftrightarrow\\
    &\Leftrightarrow& x\varphi(n+x+1) - x\varphi(n+x) = x\log(n+x) \le \varphi(n+1+x)-\varphi(n+1)
   \end{eqnarray*}
   
   Lo que lleva, usando el carácter monótono del $\log$ ($log(n) \le log(n+x) \le log(n+1)$:
   
   $$\log(n) \le \frac{\varphi(n+1+x)-\varphi(n+1)}{x} \le \log(n+1)$$
   
   Además, es fácilmente demostrable desde \ref{phi} que:
   \begin{equation}
    \varphi(n+1+x) = \varphi(x) + \log\big[x(x+1)\ldots(x+n)\big]
   \end{equation}
   
   Juntando toda esta información se tiene:
   
   $$0\le \varphi(x) - log \bigg[\frac{n!n^x}{x(x+1)\ldots (x+n)}\bigg] \le xlog\bigg(1+\frac{1}{n}\bigg)$$
   
   Además, tenemos que $\lim_{n\rightarrow \infty} log\bigg(1+\frac{1}{n}\bigg)$ para $x$ fijo, lo que nos lleva a:\\
   
   $$\varphi(x) = \Gamma(x) \lim_{n\rightarrow \infty} \bigg[\frac{n!n^x}{x(x+1)\ldots (x+n)}\bigg] $$
   
   y la existencia de dicho límite está asegurada porque $\varphi$ está definida en $]0,1[$, deduciéndose
   de a) que la igualdad es cierta para todo $x\in]0,+\infty[$
  \end{proof}

  \section{Función Beta}
  
  \begin{definition}
   Se define la función Beta con $x,y>0$ por:
   \begin{equation}
       B(x,y) = \int_0^1 {t^{x-1}(t-1)^{y-1}dt}
       \label{beta}
   \end{equation}
  \end{definition}

  
  \begin{theorem}
   Se verifica que:
    $$B(x,y) = \frac{\Gamma(x)\Gamma(y)}{\Gamma(x+y)}$$
   \end{theorem}
   
   \begin{proof}
    Fijado $y>0$ probaremos que $f(x) = \frac{\Gamma(x+y)\cdot B(x,y)}{\Gamma(y)}$ cumple las hipótesis del
    teorema \ref{caracterizacion}
    
    \begin{enumerate}[a)]
     \item Se tiene:
      \begin{eqnarray}
       B(x+1,y) &=& \int_0^1{t^x (1-t)^{y-1}dt} =\\ 	\nonumber
		&=& \int_0^1{\bigg(\frac{t}{1-t}\bigg)^x (1-t)^{x+y-1} dt} =\\	\nonumber
		&=& \bigg[\frac{(1-t)^{x+y}}{x+y}\cdot \bigg(\frac{t}{1-t}\bigg)^x\bigg]_0^1 +\\\nonumber
		&+& \frac{x}{x+y} \int_0^1{ \bigg(\frac{t}{1-t}\bigg)^{x-1} \frac{(1-t)^{x+y}}{(1-t)^2}dt}=\\		\nonumber       
		&=& \frac{x}{x+y}B(x,y)
      \end{eqnarray}
      Aplicando esto:
      \begin{eqnarray*}
      f(x+1) &=& \frac{\Gamma(x+1+y)\cdot B(x+1,y)}{\Gamma(y)} = \\
	     &=& \frac{(x+y)\Gamma(x+y)xB(x,y)}{(x+y)\Gamma(y)} = xf(x)
      \end{eqnarray*}
      
    \item $f(1) = \frac{\Gamma(1+y)B(1,y)}{\Gamma(y)} = \frac{y\Gamma(y)}{\Gamma(y)}\frac{1}{y} = 1$\\
    
      ya que $B(1,y) = y^{-1}$\\
      
    \item $log\circ B(x,y)$ es convexa para un $y$ fijo, por la desigualdad de Holder.\\
      Claramente $\log \circ f$ es convexa por ser $\Gamma$ y $B$ log-convexas (para $y$ fijo).
      
    \end{enumerate}
    \bigskip
    Lo que aplicando el Teorema \ref{caracterizacion} nos lleva a concluir $f(x) = \Gamma(x)$
   \end{proof}
   
  \subsection{Consecuencias}
  La sustitución $t=\sen^2(\theta)$ en \ref{beta} da lugar a:
  \begin{equation}
   2\int_0^{\pi/2}{(\sen\theta)^{2x-1}(\cos\theta)^{2y-1}d\theta} = \frac{\Gamma(x)\Gamma(y)}{\Gamma(x+y)}
  \end{equation}
  donde haciendo $x=y=\frac{1}{2}$ se obtiene $\Gamma\bigg(\frac{1}{2}\bigg) = \sqrt{\pi}$\\
  
  La sustitución $t=s^2$ convierte a \ref{gamma} en:
  \begin{equation}
   \Gamma(x) = 2\int_0^\infty s^{2x-1}e^{-s^2}ds \qquad x>0
  \end{equation}
  y haciendo $x=\frac{1}{2}$ se obtiene $\int_{\infty}^{\infty} e^{-s^2}ds = \sqrt{\pi}$\\
  
  Finalmente, es fácilmente deducible del Teorema \ref{caracterizacion} que:
  \begin{equation}
   \Gamma(x)= \frac{2^{x-1}}{\sqrt{\pi}}\Gamma\bigg(\frac{x}{2}\bigg)\bigg(\frac{x+1}{2}\bigg)
  \end{equation}


  

  
  %% Bibliography
  \newpage
  \begin{thebibliography}{9}

  \bibitem{algebra0}
    Walter Rudin,\\
    \emph{Chapter 8}.\\
    Principios de Análisis Matemático
    
  \end{thebibliography}

  
\end{document}