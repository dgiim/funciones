\documentclass[a4paper, 11pt]{amsart}
\usepackage{amssymb}
\usepackage[hidelinks]{hyperref}
\let\emptyset\varnothing
\listfiles

%%% Castellano
\usepackage[spanish,es-noquoting]{babel} 
\selectlanguage{spanish}
\usepackage[utf8]{inputenc}

\usepackage{amsthm}

%%% Diagramas
% Conmutative diagrams
%\usepackage{tkz-graph}
%\usetikzlibrary{arrows}
\usepackage{tikz}
\usetikzlibrary{matrix,arrows}
\tikzset{every loop/.style={min distance=10mm,in=0,out=60,looseness=10}}
\usepackage{tikz-cd}
\usetikzlibrary{cd}

%% Multicolumns
\usepackage{multicol}

%%% Shortcuts
\newcommand{\C}{\mathcal{C} }

\usepackage{enumerate}
%\usepackage{enumitem}
%%% Twopart definition
\newcommand{\twopartdef}[4]
{
	\left\{
		\begin{array}{ll}
			#1 & \mbox{si } #2 \\
			#3 & \mbox{si } #4
		\end{array}
	\right.
}



\newtheorem{theorem}{Teorema}[section]
\newtheorem{lemma}[theorem]{Lema}

\theoremstyle{definition}
\newtheorem{definition}[theorem]{Definición}
\newtheorem{example}[theorem]{Ejemplo}
\newtheorem{exca}[theorem]{Ejercicio}

\theoremstyle{remark}
\newtheorem{remark}[theorem]{Remark}

\numberwithin{equation}{section}



%%  DIAGRAM MACROS
\usepackage{xparse}
\usepackage{etoolbox}
\newcounter{listtotal}\newcounter{listcntr}%
\NewDocumentCommand{\argument}{o}{%
  \setcounter{listtotal}{0}\setcounter{listcntr}{-1}%
  \renewcommand*{\do}[1]{\stepcounter{listtotal}}%
  \expandafter\docsvlist\expandafter{\argumentarray}%
  \IfNoValueTF{#1}
    {\namesarray}% \names
    {% \names[<index>]
     \renewcommand*{\do}[1]{\stepcounter{listcntr}\ifnum\value{listcntr}=#1\relax##1\fi}%
     \expandafter\docsvlist\expandafter{\argumentarray}}%
}

%% Product Object
% 1-9:  A, B, Z, AxB, pa, pb, f, g, fxg
%%
\newcommand{\productCD}[1]{
  \def\argumentarray{#1}
  \begin{tikzcd}[column sep=tiny, ampersand replacement=\&]
      \& \argument[2] \arrow[dashed]{d}[description]{\argument[8]} \arrow[bend left]{ddr}{\argument[7]} \arrow[bend right]{ddl}[swap]{\argument[6]} \& \\
      \& {\argument[3]} \arrow{dl}{\argument[4]} \arrow{dr}[swap]{\argument[5]} \& \\
    \argument[0] \& \& \argument[1]
  \end{tikzcd}
}

%% Coproduct Object
% 1-9:  A, B, Z, AUB, ia, ib, f, g, (f,g)
%%
\newcommand{\coproductCD}[1]{
  \def\argumentarray{#1}
  \begin{tikzcd}[column sep=tiny, ampersand replacement=\&]
      \& \argument[2] \& \\
      \& {\argument[3]} \arrow[dashed]{u}[description]{\argument[8]} \& \\
    \argument[0] \arrow[bend left]{uur}{\argument[6]} \arrow{ur}[swap]{\argument[4]}  \& \& \argument[1] \arrow[bend right]{uul}[swap]{\argument[7]} \arrow{ul}{\argument[5]}
  \end{tikzcd}
}

\everymath{\displaystyle} 

\begin{document}

\title{Funciones con nombre propio}

%    Remove any unused author tags.

%    author one information
\author{Ignacio Cordón}
\address{}
\curraddr{}
\email{}
\thanks{}

% References
%\subjclass[2000]{Primary }
%    For articles to be published after 1 January 2010, you may use
%    the following version:
%\subjclass[2010]{Primary }

\keywords{}

\date{}

\dedicatory{}


\begin{abstract}
  Definición y propiedades fundamentales de algunas funciones con nombre propio.
\end{abstract}

\maketitle

  \section{Función gamma}
  
  \definition
  Para $0 < x < \infty$, se define la funcion gamma como: 
      $$\Gamma(x) = \int_0^{\infty}{t^{x-1}e^{-t}dt}$$
      
  Es sabido que esta integral converge para cualquier valor de $x\in]0,+\infty[$, y que por tanto la definición de 
  $\Gamma$ es correcta.\\
  
  \begin{theorem}
    Se verifica:
      \begin{enumerate}[(a)]
	\item $\Gamma(x+1) = \Gamma(x)$
	\item $\Gamma(n+1) = n!$ para cualquier $n\in \mathbb{N}$
	\item $\log \circ \Gamma$ es convexa en $]0,+\infty[$
      \end{enumerate}
  \end{theorem}
  
   \begin{proof}
    \\\
     \begin{enumerate}[(a)]
      \item Integrando por partes tenemos:
	\begin{eqnarray*}
	\Gamma(x+1) &=& \int_0^{\infty}{t^{x}e^{-t}dt} = \bigg[-e^{-t}t^x\bigg]_0^{\infty} +
	  \int_0^{\infty}{xt^xe^{-t}dt} = \\
	  & = & x\int_0^{\infty}{t^{x-1}e^{-t}dt} = x\Gamma(x)
	\end{eqnarray*}
      \item Inducción sobre $n$, usando (a) y $\Gamma(1) = \int_0^{\infty}{e^{-t}dt} = 
	\bigg[-e^{-t}\bigg]_0^{\infty} = 1 $
      \item Dados $p,q \in \mathbb{N} : \frac{1}{p} + \frac{1}{q} = 1$\\
      
	La \textit{desigualdad de Young} para $a,b \ge 0$ nos dice: 
	  $$ab \le \frac{a^p}{p} + \frac{b^q}{q}$$\\
	La \textit{desigualdad de Holder} nos dice: 
	  $$\bigg|\int_a^b {fg dt}\bigg| \le 
	  \bigg(\int_a^b {|f|^p dt}\bigg)^{1/p} \bigg(\int_a^b {|g|^p dt}\bigg)^{1/q}$$
	  
	Así:
	
	\begin{eqnarray*}
	 \Gamma\bigg(\frac{x}{p} +\frac{y}{q}\bigg) & = & 
	    \int_0^{\infty}{t^{\big(\frac{x}{p}+\frac{y}{q}
	    -\frac{1}{p}-\frac{1}{q}\big)}e^{-t}dt} 
	    =\int_0^{\infty}{t^{\big(\frac{1}{p}(x-1)\big)} t^{\frac{1}{q}(y-1)\big)}e^{-t}dt} =\\
	    \\
	 & = &_{(*)} \le \Gamma(x)^{1/p}\Gamma(x)^{1/q}
	\end{eqnarray*}\\
	
	Donde en $(*)$ se ha usado la desigualdad de Holder anteriormente descrita.

     \end{enumerate}
      
    \end{proof}
  
  De hecho estas tres propiedades caracterizan a la función $\Gamma$, como demostraron Bohr y
  Mollerup en el teorema al que dan nombre:
      
  \begin{theorem}{Caracterización de Bohr-Mollerup}\\
  
  Sea $f\ge 0$ sobre $]0,+\infty[$ verificando:
  
  \begin{enumerate}[(a)]
   \item $f(x+1) = xf(x)$
   \item $f(1) = 1$
   \item $\log \circ f es convexa$
  \end{enumerate}\\
  
   Entonces $f(x) = \Gamma(x)$
   
  \end{theorem}

      

  
  %% Bibliography
  \vfill
  \begin{thebibliography}{9}

  \bibitem{algebra0}
    Walter Rudin,\\
    \emph{Chapter 8}.\\
    Principios de Análisis Matemático
    
  \end{thebibliography}

  
\end{document}